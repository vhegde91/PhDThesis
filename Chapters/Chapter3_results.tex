\section{Results \& statistical interpretations}
\label{sec:results}

The background predictions described in Section~\ref{sec:bkgestimation} are 
summed together for the final statistical SMS interpretations.   This is
done use the Higgs combine framework~\cite{Hcombine}, where each analysis
bin is represented as a individual counting experiment with one 
signal process and five background processes.  Each process is assumed 
to be a Poisson distribution with some number of prior distribution representing
the uncertainty on the corresponding process's rate parameter. The exact
details of how these uncertainties are modeled depends on the 
details of each of the background estimation methods that were used. 

In general, each background estimation is a scaling of a Poisson process 
from a control region, $N^{\mathrm{pred}}_{\mathrm{bkg}}=N^{\mathrm{pred}}_{\mathrm{CR}}\beta$, where $\beta$ 
is the transfer factor, $N^{\mathrm{pred}}_{\mathrm{CR}}$ is the expected yield in the 
control region of the corresponding process, and $N^{\mathrm{pred}}_{\mathrm{bkg}}$ is the 
prediction provided as the expected mean of the corresponding process's
Poisson distribution for a given signal region.  In general,  $N^{\mathrm{pred}}_{\mathrm{CR}}$
is constrained by the observed yields in the control region, 
$N^{\mathrm{obs}}_{\mathrm{CR}}$ and there are five such constraints from each 
of the 5 control regions discussed in Section~\ref{sec:bkgestimation}.  For a 
simplified counting experiment with only one background process and one signal
process, the likelihood for this type of background prediction would be, 

\begin{equation}
\centering
\mathcal{L}(N^{\mathrm{pred}}_{\mathrm{sig}},N^{\mathrm{pred}}_{\mathrm{bkg}}|N^{\mathrm{obs}}_{\mathrm{SR}},N^{\mathrm{obs}}_{\mathrm{CR}}) \propto (N^{\mathrm{pred}}_{\mathrm{bkg}}+N^{\mathrm{pred}}_{\mathrm{sig}})^{N^{\mathrm{obs}}_{\mathrm{SR}}}e^{-N^{\mathrm{pred}}_{\mathrm{bkg}}}\cdot(\frac{N^{\mathrm{pred}}_{\mathrm{bkg}}}{\beta})^{N^{\mathrm{obs}}_{\mathrm{CR}}} e^{-N^{\mathrm{pred}}_{\mathrm{bkg}}/\beta}
\end{equation}

This likelihood, is equivalent to a Poisson likelihood with a gamma distribution
as the prior and can be used to constrain both the predicted background the 
signal strength in the signal region with the proper modeling of the $N^{\mathrm{pred}}_{\mathrm{bkg}}$ 
uncertainty due to limited statistics of the control region observed yield.

As such, we modeling the uncertainty on the prediction due to limited 
control region statistics as a gamma distribution.  This procedure is the 
recommended procedure of the CMS Statistics Committee, especially when the 
observed control region statistics are too small to justifying a Gaussian 
approximation.  While the observed control region events vary considerably, 
for the signal regions in extreme corners of our phase space the use of a 
Gamma prior distribution is critical for modeling uncertainties.  For uniformity, 
we generally use it everywhere.  The exception to this, as mentioned below, is 
the $e$+jets control, 
which is used for estimating the fake-photon background, where we use a Gaussian prior 
to model the effect of limited control region statistics.  Note, the number of 
observed events in these control regions is always larger than 17 events, for 
which the $1\sigma$ intervals of a Poisson disribution and Gaussian distribution
still agree to within ~10\%.  

%While the uncertainty due to $N^{\mathrm{obs}}_{\mathrm{CR}}$ is typically modeled with 
%a gamma distribution, 
The uncertainty on the transfer factors used to scale the observed event yields 
in the control regions are modeled with log-normal prior
distributions.  These priors account for uncertainties typically arising 
from limited MC statistics, systematic effects do to composition or limited knowledge 
of object reconstruction efficiencies, or PDF/scale uncertainties.  Uncertainties 
from limited MC statistics are typically uncorrelated across each of the 
search regions.  Other source of systematic uncertainty typically correlate 
expected yields from various signal regions.  Details of the modeling of 
systematic uncertainties for each of the background estimations is given below. 

For the lost-e and lost-$\mu+\tau_{\mathrm{had}}$ predictions, the uncertainties
on the prediction are modeled with a gamma distribution, with the 
shape parameter set to the observed number of events in the corresponding
$\ell-\gamma$ control region and the scale parameter is set to be the
average transfer factor, given in Table~\ref{tab:lost_lepton_SF}.  There 
are also log-normal prior distributions used to model the uncertainties
of the transfer factors, which account for:

\begin{itemize}
 \item limited MC statistics; the corresponding nuisance parameters are fully uncorrelated
 \item PDF and scale uncertainties; the corresponding nuisance parameters are fully correlated across signal regions and between $e+\gamma$ and $\mu+\gamma$ control regions
 \item jet energy scale uncertainties; the corresponding nuisance parameters are fully correlated across all signal regions
 \item lepton scale factor uncertainties; the corresponding nuisance parameters are fully correlated across \ptmiss bins and fully uncorrelated across \nb, \nj regions;
 \item uncertainties related to soft/collinear photon modeling; the corresponding nuisance parameters are fully correlated (negligible for $\mu+\tau_{\mathrm{had}}$ prediction).
 %\item b-tagging scale factor uncertainties
\end{itemize}
For the fake-rate estimation of $W+\gamma$ and $t\bar{t}+\gamma$ events, the systematics
from the control region statistics are typically small.  All of the uncertainties
are modeled with log-normal prior distributions whose uncertainties
correspond to:

\begin{itemize}
 \item the statistical uncertainty from the $e\rightarrow\gamma$ control regions; the corresponding nuisance parameters are fully uncorrelated across all signal regions;
 \item fake rate scale factors; the corresponding nuisance parameters are fully correlated across all signal regions; 
 \item uncertainties due to pileup and ISR modeling of simulations; the corresponding nuisance parameters are fully correlated across all signal regions.
\end{itemize}

For the Z$\gamma$ predictions, the statistical uncertainties from control 
region are modeled with a gamma
prior distribution whose shape parameter is the observed events in the 
$Z(\ell\ell)\gamma$ control region and the scale parameter is the 
purity-corrected MC transfer factor, which in the nomenclature of 
Equation~\ref{eq:zgamma} is 
$\beta\cdot N_{\nu\nu+\gamma}^{\mathrm{MC}}/N_{\ell\ell+\gamma}^{\mathrm{MC}}$.
Additional log-normal prior distributions are included to account 
for

\begin{itemize}
 \item the effect of limited MC statistics on $N_{X+\gamma}^{\mathrm{MC}}$; the corresponding nuisance parameters are fully uncorrelated across all signal regions;
 \item uncertainties on the purity; the corresponding nuisance parameter fully correlates the effect across all signal regions;
 \item uncertainties due to b-tag scale factors; the corresponding nuisance parameters are fully correlated across \ptmiss and \nj bins, anti-correlated across \nb regions;
 \item uncertainty due to missing higher order corrections; the corresponding nuisance parameters are fully correlated across \ptmiss bins, uncorrelated across \nb, \nj regions.
\end{itemize}


For the multijet predictions, the systematic uncertainties due to limited 
control region statistics are modeled with a 
Gamma prior distribution whose shape parameter is the observed number
of events in the low-$\Delta\Phi$ control regions and the scale parameter
is the product of the high-to-low ratio, $R_{\mathrm{h/l}}$, the double ratio
correction factors, K, and the purity, $\beta$, obtained from data-driven predictions
of the electroweak backgrounds in the low-$\Delta\Phi$ control regions. 
Systematic uncertainties on the knowledge of this combined transfer factor are 
modeled with log-normal prior distributions and account for

\begin{itemize}
 \item uncertainties on $R_{\mathrm{h/l}}$ due to limited number of events in data sideband; the corresponding nuisance parameters 
 are fully correlated across \ptmiss bins and fully uncorrelated across \nb, \nj regions;
 \item uncertainties on K based on validations with zero photon events; the corresponding nuisance parameters are fully 
   correlated across \ptmiss bins and fully uncorrelated across \nb and \nj regions;
 \item uncertainties on $\beta$ due to all uncertainties as described above; the corresponding nuisance parameters are 
  fully uncorrelated across all search regions.
\end{itemize} 

All of the uncertainties associated with the signal yields, described in detail
in Section~\ref{sec:signal-systematics}, are modeled with log-normal prior distributions.  
Nuisance parameters associated with ISR modeling uncertainties are correlated 
across the various \nj bins, but uncorrelated across \nb and \ptmiss regions.  
Nuisance parameters associated with \ptmiss modeling in simulation are
correlated across the various \ptmiss bins and uncorrelated across \nb and 
\nj regions.  Nuisance parameters associated with b-tagging scale factors are fully correlated
across all \ptmiss and \nj, but anticorrelated across the \nb regions. The nuisance 
parameters associated with the statistical uncertainties of our simulated samples
are fully uncorrelated across all signal regions. A single nuisance parameter
is associated with the luminosity uncertainty and fully correlates the effect
across all signal regions. 

Expected and observed limits are computed after adjusting the central values
and uncertainties associated with all nuisance parameters to data by minimizing 
them with respect to the observed yields while fixing the signal strength to
zero. A detailed comparison of all nuisance parameters before (prefit) and after
(postfit) this fit to data is presented in Appendix~\ref{sec:prefit_postfit}.
%Also in Appendix~\ref{sec:prefit_postfit} is a detailed enumeration of all 
%nuisance parameters and their correlation assumptions.   

All of the predictions and the total systematic uncertainties associated with 
them are show in Tables~\ref{tab:finalPrediction0b} and~\ref{tab:finalPrediction1b}.
Figure~\ref{fig:predVsData} shows a stack plot of the predictions in the 25 search bins. Note, 
the low \ptmiss sideband (100-200~\gev) is excluded from the tables and figures since it, by construction,
has perfect data/MC agreement. A large discrepancy is observed in SR bin 2 where 51 events are observed while 
a total of 89 background events are expected. More diagnostics ../Figures/Chap3 in this bin and 
adjacent bins are further scrutinized in Figure \ref{fig:sr123-1} of Appendix~\ref{sec:sr-bin2-diagonostics}.
We also checked similar distributions in single electron and low-\dphi control regions
used for estimating fake photon and fake \ptmiss backgrounds in figures~\ref{fig:sr123-ele} and \ref{fig:sr123-ldp} respectively. 
We do not observe any problematic features in these distributions and shapes generally match well between data and MC.

\begin{table}[htbp]
%\renewcommand{\arraystretch}{1.25}
\centering
\caption{Predicted and observed event yields for each of the $\nb$ = 0 signal regions.}
\label{tab:finalPrediction0b}
\begin{tabular}{cccccccc}
%\hline \hline 
\ptmiss [GeV] & Z$\gamma$ & Lost-e & Lost-$\mu+\tau_{\mathrm{had}}$ & $e\rightarrow\gamma$ & Multijet & Total & Data \\\hline
\multicolumn{8}{c}{$2\leq\nj\leq 4, \nb=0$} \\\hline
$200-270$ & 33.6 $\pm$ 8.3    & 10.5 $\pm$ 2.6   & 31.2 $\pm$ 6.0    & 22.3 $\pm$ 5.4    & 60   $\pm$ 11   & 157 $\pm$ 16  & 151 \\
$270-350$ & 22.9 $\pm$ 6.0    & 5.8  $\pm$ 1.8   & 29.6 $\pm$ 5.9    & 11.9 $\pm$ 2.9    & 20.5 $\pm$ 4.3  & 91  $\pm$ 10  & 51 \\
$350-450$ & 17.0 $\pm$ 5.2    & 1.68 $\pm$ 0.88  & 13.9 $\pm$ 3.9    & 6.6 $\pm$ 1.6     & 4.1  $\pm$ 1.4  & 43.3$\pm$ 6.8 & 50 \\
$450-750$ & 18.1 $\pm$ 7.1    & 1.98 $\pm$ 0.94  & 8.1  $\pm$ 3.1    & 6.7 $\pm$ 1.5     & 2.5  $\pm$ 1.3  & 37.4$\pm$ 8.0 & 33 \\        
$>750$     & 2.8 $\pm$ 1.2    & $0.00_{-0.00}^{+0.69}$  & 1.2 $\pm$ 1.2  & 0.79 $\pm$ 0.19  & $0.41_{-0.41}^{+0.42}$  & 5.2 $\pm$ 1.9  & 6 \\
\hline 
\multicolumn{8}{c}{$5\leq\nj\leq 6, \nb=0$} \\\hline
$200-270$ & 3.09 $\pm$ 0.78  & 1.28 $\pm$ 0.61  & 5.1  $\pm$ 1.9     & 3.53 $\pm$ 0.75  & 15.8 $\pm$ 4.8  & 28.8 $\pm$ 5.3  & 26 \\
$270-350$ & 1.98 $\pm$ 0.54  & 2.06 $\pm$ 0.80  & 3.2  $\pm$ 1.5     & 2.39 $\pm$ 0.56  & 3.7  $\pm$ 1.8  & 13.3 $\pm$ 2.6  & 11  \\
$350-450$ & 1.49 $\pm$ 0.47  & 0.77 $\pm$ 0.46  & $0.64_{-0.64}^{+0.65}$  & 1.26 $\pm$ 0.30  & 1.23 $\pm$ 0.97  & 5.4 $\pm$ 1.4  & 8 \\
$>450$ & 1.65 $\pm$ 0.65  & 0.26 $\pm$ 0.26  & 1.9 $\pm$ 1.1  & 1.00 $\pm$ 0.24  & $0.07_{-0.07}^{+0.52}$  & 4.9 $\pm$ 1.4   & 7 \\
\hline 
\multicolumn{8}{c}{$\nj\geq 7, \nb=0$} \\\hline
$200-270$ & 0.37 $\pm$ 0.11  & $0.00_{-0.00}^{+0.61}$  & $0.0_{-0.0}^{+1.3}$  & 0.72 $\pm$ 0.16  & 1.8 $\pm$ 1.2  & 2.9 $\pm$ 1.9  & 3 \\
$270-350$ & 0.24 $\pm$ 0.08  & $0.34_{-0.34}^{+0.35}$  & 1.5 $\pm$ 1.0  & 0.38 $\pm$ 0.10  & 1.22 $\pm$ 0.94  & 3.6 $\pm$ 1.5  & 3 \\
$350-450$ & 0.16 $\pm$ 0.07  & $0.34_{-0.34}^{+0.35}$  & 0.73 $\pm$ 0.73  & 0.17 $\pm$ 0.05  & $0.07_{-0.07}^{+0.50}$  & 1.46 $\pm$ 0.96  & 0 \\
$>450$ & 0.17 $\pm$ 0.08  & $0.00_{-0.00}^{+0.61}$  & $0.0_{-0.0}^{+1.3}$  & 0.20 $\pm$ 0.06  & $0.00_{-0.00}^{+0.75}$  & $0.37_{-0.37}^{+1.60}$  & 0 \\
\end{tabular}
\end{table}

\begin{table}[htbp]
%\renewcommand{\arraystretch}{1.25}
\centering
\caption{predicted and observed event yields for each of the $\nb\geq1$ signal regions.}
\label{tab:finalPrediction1b}
\begin{tabular}{cccccccc}
%\hline \hline 
\ptmiss [GeV] & Z$\gamma$ & Lost-e & Lost-$\mu+\tau_{\mathrm{had}}$ & $e\rightarrow\gamma$ & Multijet & Total & Data \\\hline
\multicolumn{8}{c}{$2\leq\nj\leq4, \nb\geq1$} \\\hline
$200-270$ & 3.55 $\pm$ 0.89  & 3.4  $\pm$ 1.5   & 14.5 $\pm$ 4.2   & 7.1  $\pm$ 1.7   & 11.3 $\pm$ 3.3   & 39.8  $\pm$ 5.9   & 50 \\
$270-350$ & 2.45 $\pm$ 0.65  & 2.9  $\pm$ 1.4   & 5.6  $\pm$ 2.5   & 3.79 $\pm$ 0.92  & 5.7  $\pm$ 1.8   & 20.4  $\pm$ 3.6   & 20   \\
$350-450$ & 1.81 $\pm$ 0.55  & $0.0_{-0.0}^{+1.0}$  & 1.1 $\pm$ 1.1  & 2.00 $\pm$ 0.45  & 0.59 $\pm$ 0.44  & 5.5  $\pm$ 1.7  & 4  \\
$>450$ & 2.14 $\pm$ 0.84  & 2.3 $\pm$ 1.2  & 4.4 $\pm$ 2.3  & 1.62 $\pm$ 0.38  & 0.95 $\pm$ 0.54  & 11.5 $\pm$ 2.8  & 8  \\
\hline
\multicolumn{8}{c}{$5\leq\nj\leq6, \nb\geq1$} \\\hline
$200-270$ & 0.76 $\pm$ 0.20  & 3.5  $\pm$ 1.3   & 2.4  $\pm$ 1.4   & 5.5  $\pm$ 1.2   & 7.7  $\pm$ 2.4   & 19.9  $\pm$ 3.3   & 21 \\
$270-350$ & 0.49 $\pm$ 0.14  & 1.06 $\pm$ 0.64  & 4.0  $\pm$ 1.8   & 2.98 $\pm$ 0.63  & 2.1 $\pm$ 1.0  & 10.6  $\pm$ 2.3  & 15 \\
$350-450$ & 0.32 $\pm$ 0.11  & 0.71 $\pm$ 0.51  & 2.4  $\pm$ 1.4   & 1.38 $\pm$ 0.29  & $0.30_{-0.30}^{+0.49}$  & 5.1 $\pm$ 1.6  & 6   \\
$>450$ & 0.48 $\pm$ 0.20  & $0.35_{-0.35}^{+0.36}$  & $0.0_{-0.0}^{+1.4}$  & 0.67 $\pm$ 0.15  & $0.00_{-0.00}^{+0.56}$  & $1.5_{-1.5}^{+1.6}$  & 2 \\
\hline
\multicolumn{8}{c}{$\nj\geq7, \nb\geq1$} \\\hline
$200-270$ & 0.13 $\pm$ 0.04  & 0.72 $\pm$ 0.53  & 2.0 $\pm$ 1.2  & 1.68 $\pm$ 0.37  & 5.9 $\pm$ 5.0  & 10.5 $\pm$ 5.1  & 12 \\
$270-350$ & 0.10 $\pm$ 0.04  & $0.00_{-0.00}^{+0.65}$  & 1.33 $\pm$ 0.96  & 0.73 $\pm$ 0.16  & $0.0_{-0.0}^{+1.1}$  & 2.2 $\pm$ 1.6  & 1 \\
$350-450$ & 0.07 $\pm$ 0.03  & 0.72 $\pm$ 0.53  & $0.0_{-0.0}^{+1.2}$  & 0.44 $\pm$ 0.10  & $0.0_{-0.0}^{+1.1}$  & $1.2_{-1.2}^{+1.7}$  & 1 \\
$>450$ & 0.04 $\pm$ 0.02  & $0.36_{-0.36}^{+0.37}$  & $0.0_{-0.0}^{+1.2}$  & 0.23 $\pm$ 0.07  & $0.0_{-0.0}^{+1.1}$  & $0.6_{-0.6}^{+1.7}$  & 1 \\
\end{tabular}
\end{table}

\begin{figure}
\centering
\includegraphics[width=0.98\linewidth]{../Figures/Chap3/results/c_predVsDataSBins.pdf}
\caption{Distribution of predicted background events and observed events in each of the 25 signal regions.}
\label{fig:predVsData}
\end{figure}

The test statistic $q_{\mu} = -2 \ln{\mathcal{L}_{\mu}/ \mathcal{L}_{\mathrm{max}}}$, 
where $\mathcal{L}_{\mathrm{max}}$ 
is the maximum likelihood determined by allowing all parameters, including the signal 
strength, to float, and $\mathcal{L}_{\mu}$
is the profiled likelihood. Limits are determined using the asymptotic form of the test 
statistic \cite{Cowan:2010js} 
in conjunction with the $CL_s$ criterion \cite{Junk1999,bib-cls}. Expected 
upper limits are derived by varying observed yields according to expectations from the 
background only hypothesis.

Using the statistical procedure described above, 95\% confidence level (CL) upper limits are
computed on the signal cross section for each simplified model and each mass point.  
Figure~\ref{fig:T5bbbbExclusion} shows the expected and observed exclusion contours for the 
T5bbbbZG and T5bbbbHG SMS 
models in the gluino-NLSP mass plane.  The largest gluino exclusion limits occur for 
points with moderate NLSP mass, where the LSP, SM bosons, and the b-quarks from
gluino decays are imparted with large momentum; we exclude gluino 
masses below 2120~\gev.  For small NLSP masses, limits are degraded because
photon \pt and \ptmiss spectrums become softer, lowering the signal efficiency;
we exclude gluino masses below 1940~\gev.

For very low NLSP masses ($<$ 91\gev), in
case of T5bbbbZG and T5ttttZG models, sensitivity starts to increase if NLSP mass is
below Z mass. In these cases, Z is off-shell and gravitino carries more momentum gives larger \ptmiss.
So the sensitivity increases and hence a kink is observed in the contours. Acceptance $\times$ efficiency plots
shown in figure \ref{fig:sigEff} in Appendix \ref{subsec:SigAccEff} also support this argument.
In T5qqqqHG models, there are no mass points below Higgs mass since Higgs is always on shell.

For large NLSP masses the limits are degraded because the jet multiplicity and b-tagged jet multiplicity
is reduces, and in turn ST is lowered.  This has the effect that the overall 
signal efficiency is degraded and signal events tend to be concentrated in bins
with higher background yields, as compared to the moderate NLSP mass points.  
For significance calculations we use the profile-likelihood calculation with one times signal 
expectation included.  Figure~\ref{fig:T5bbbbSignficance} - \ref{fig:T6ttSignficance} show the observed significance
for each point in the gluino-NLSP mass plane for the T5qqqqHG, T5bbbbZG, T5ttttZG and T6ttZG SMS models.

\begin{figure}[h]
\centering
\includegraphics[width=0.66\linewidth]{../Figures/Chap3/results/T5qqqqHg_exclusion.pdf}
\caption{Expected and observed exclusion contours in the $m_{\tilde{g}}-m_{\tilde{\chi}_{1}^{0}}$ plane for the T5qqqqHG SMS model.}
\label{fig:T5qqqqExclusion}
\end{figure}

\begin{figure}[h]
\centering
\includegraphics[width=0.66\linewidth]{../Figures/Chap3/results/T5bbbbZg_exclusion.pdf}
\caption{Expected and observed exclusion contours in the $m_{\tilde{g}}-m_{\tilde{\chi}_{1}^{0}}$ plane for the T5bbbbZG SMS model.}
\label{fig:T5bbbbExclusion}
\end{figure}

\begin{figure}[h]
\centering
\includegraphics[width=0.66\linewidth]{../Figures/Chap3/results/T5ttttZg_exclusion.pdf}
\caption{Expected and observed exclusion contours in the $m_{\tilde{g}}-m_{\tilde{\chi}_{1}^{0}}$ plane for the T5ttttZG SMS model.}
\label{fig:T5ttttExclusion}
\end{figure}

\begin{figure}[h]
\centering
\includegraphics[width=0.66\linewidth]{../Figures/Chap3/results/T6ttZg_exclusion.pdf}
\caption{Expected and observed exclusion contours in the $m_{\tilde{t}}-m_{\tilde{\chi}_{1}^{0}}$ plane for the T6ttZG SMS model.}
\label{fig:T6ttExclusion}
\end{figure}


\begin{figure}[h]
\centering
%\includegraphics[width=0.49\linewidth]{../Figures/Chap3/results/T5qqqqHg_ExpSignif.pdf}
\includegraphics[width=0.49\linewidth]{../Figures/Chap3/results/T5qqqqHg_ObsSignif.pdf}
\caption{Observed significance in the $m_{\tilde{g}}-m_{\tilde{\chi}_{1}^{0}}$ plane for 
 T5qqqqHG SMS model.}
%The color map shows the expected significance for each mass point.  The red line shows the contour for which we expect $5\sigma$ expected significance. Right plot shows the observed significance. For $m_{\tilde{\chi}_{1}^{0}}$ below 1000\gev, observed significance is consistent with 0.}
\label{fig:T5qqqqSignficance}
\end{figure}

\begin{figure}[h]
\centering
%\includegraphics[width=0.49\linewidth]{../Figures/Chap3/results/T5bbbbZg_ExpSignif.pdf}
\includegraphics[width=0.49\linewidth]{../Figures/Chap3/results/T5bbbbZg_ObsSignif.pdf}
\caption{Observed significance in the $m_{\tilde{g}}-m_{\tilde{\chi}_{1}^{0}}$ plane for
 T5bbbbZG SMS model.}
%The color map shows the expected significance for each mass point.  The red line shows the contour for which we expect $5\sigma$ expected significance. Right plot shows the observed significance.}
\label{fig:T5bbbbSignficance}
\end{figure}

\begin{figure}[h]
\centering
%\includegraphics[width=0.49\linewidth]{../Figures/Chap3/results/T5ttttZg_ExpSignif.pdf}
\includegraphics[width=0.49\linewidth]{../Figures/Chap3/results/T5ttttZg_ObsSignif.pdf}
\caption{Observed significance in the $m_{\tilde{g}}-m_{\tilde{\chi}_{1}^{0}}$ plane for
 T5ttttZG SMS model.}
%The color map shows the expected significance for each mass point.  The red line shows the contour for which we expect $5\sigma$ expected significance. Right plot shows the observed significance.}
\label{fig:T5ttttSignficance}
\end{figure}

\begin{figure}[h]
\centering
%\includegraphics[width=0.49\linewidth]{../Figures/Chap3/results/T6ttZg_ExpSignif.pdf}
\includegraphics[width=0.49\linewidth]{../Figures/Chap3/results/T6ttZg_ObsSignif.pdf}
\caption{Observed significance in the $m_{\tilde{t}}-m_{\tilde{\chi}_{1}^{0}}$ plane for
 T6ttZG SMS model.}
%The color map shows the expected significance for each mass point.  The red line shows the contour for which we expect $5\sigma$ expected significance. Right plot shows the observed significance.}
\label{fig:T6ttSignficance}
\end{figure}
