\chapter{Introduction} % Main chapter title

\label{Chap0}
%----------------------------------------------------------------------------------------
Since time immemorial, human beings have wondered about the causes of natural phenomena taking place in the universe.
A persistent question has been about the building blocks of the universe and what holds it together.
Based on our understanding, as of today, there are four fundamental forces:
\begin{itemize}
\item Gravitational force: this causes apples to fall and also the planets to revolve around the sun.
\item Electromagnetic force: holds atoms together, basic principle in chemistry. Almost everything, except gravity, that we come across in daily life is governed by this force.
\item Weak force: $\beta$ decay taking place in the nucleus is because of this force.
\item Strong force: the force which holds protons and neutrons together in the nucleus.
\end{itemize}
The matter that we see around us is made up of atoms. An atom is made up of a nucleus (consisting of protons and neutrons, collectively 
called nucleons) and electrons. Nucleons are composed of fundamental particles called quarks. So the matter around us can be thought of as 
some combination of electrons and quarks, mainly. Apart from these, there are also other fundamental particles, some of which are similar \
to electrons and quarks but not the dominant constituents of ordinary matter, some are mediators of the fundamental forces and one
is responsible for particles to acquire mass. There are 3 flavors of charged leptons, 3 flavors of neutral leptons, $3\times2$ flavors of 
quarks, 5 mediators of forces and a Higgs boson. In total there are 18 such fundamental particles which we know today. Can we put all these 
particles and forces into a framework and try to understand the phenomena around us? The attempt that we made so far for this purpose has 
led to the standard model (SM) of particle physics \cite{PhysRevLett.19.1264,RevModPhys.52.525,GLASHOW1961579}. Our attempt is not 
complete because in the framework of SM, we could not fit gravitational force or interaction and also the mediator of gravity, the 
graviton. Graviton is a hypothetical particle, not experimentally detected, not included in the SM. So the SM consists of 17 fundamental 
particles and their mutual interactions can be explained by three fundamental forces.

The SM has successfully explained many of the experimental results; with very high degree of precision in certain cases, such as 
the measurement of magnetic moment of electron \cite{PhysRevLett.97.030801}. But the SM is not 
a complete story of the universe. %, one of the simple reason being, not having gravity included in it. 
There are many issues in the SM, such as the inability to explain dark matter \cite{Zwicky:1933gu,Clowe:2006eq} and dark energy 
\cite{Riess:1998cb} in the universe, 
matter anti-matter asymmetry \cite{doi:10.1146/annurev.aa.14.090176.002011,Canetti:2012zc} etc, and many theoretical problems such as 
hierarchy problem \cite{tHooft:1979rat}, imperfect gauge coupling unification etc. To address some of these questions, many 
extensions of SM have been proposed, one of them being the supersymmetry (SUSY). SUSY predicts that for every SM particle, there is a 
superpartner similar in all the quantum numbers and properties except that the spin quantum number differs by half a unit. Failure to 
experimentally observe supersymmetric particles with same mass as SM particles (we did not find supersymmetric electron, 
called selectron or supersymmetric proton, called sproton etc), led to the idea that SUSY is broken and the sparticles are massive as compared to their SM 
counterparts. There are many ways of breaking SUSY, one of them being gauge mediated SUSY breaking 
(GMSB). In scenarios with GMSB, gravitino (the super partner of hypothetical graviton) is the lightest supersymmetric particle (LSP) and it is 
stable. The stability of LSP is a requirement from R-parity conservation which does not allow supersymmetric particle decay into only SM 
particles. LSP is weakly interacting particle and hence it does not leave any visible signature in the detector and it is a viable dark 
matter candidate.

The compact muon solenoid (CMS) detector at the large hadron collider (LHC), CERN is a multipurpose detector designed for
testing the SM and also to carry out searches for phenomena taking place beyond SM (if there are any such phenomena),
such as supersymmetric particle production at $\sim$ TeV scale. At the LHC, proton-proton (pp) collisions 
take place at the center of mass energy of 13 TeV. Supersymmetric particles can be produced in these collisions if they exist and 
accessible at the LHC. In GMSB models, events with one or more photons, quarks of light flavor (up, down, charm and strange) or bottom 
flavor and missing transverse momentum are expected. Missing transverse momentum is the signature of weakly interacting LSPs (or 
neutrinos) in the detector which is nothing but the momentum imbalance in directions transverse to the colliding beams.

This thesis investigates the production of supersymmetric particles with at least one photon, light flavor  
or bottom flavor jets which originate from quarks and large 
missing transverse momentum using the data collected in the year 2016 with the CMS detector. Chapter \ref{Chap1} gives a very brief 
introduction to SM, its limitations and supersymmetric extensions of SM and GMSB models. Chapter \ref{Chap2} describes the experimental 
set up consisting of LHC and the CMS detector. I also discuss a new method to inter-calibrate short fibers of forward hadron calorimeter in this chapter.  The 
analysis strategy and background estimation techniques are discussed in chapter \ref{Chap3}. Results, summary and some possible extensions 
of the search are described in the final chapter \ref{Chap4}.